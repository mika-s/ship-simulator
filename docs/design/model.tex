\documentclass[a4paper]{article}  

\usepackage{amsmath}
\usepackage{amssymb}
\usepackage{mathrsfs}
\usepackage[toc,page]{appendix}
\usepackage[pdftex]{graphicx}
\usepackage{geometry}
\usepackage[utf8]{inputenc}
\usepackage{tabularx}
\usepackage{color}
\usepackage{natbib}
\usepackage{epstopdf}
\usepackage{caption}
\usepackage{subcaption}
\usepackage{cancel}
\usepackage{pdfpages}

\bibliographystyle{plainnat}
\newcommand{\HRule}{\rule{\linewidth}{1mm}}
\newcommand{\myparagraph}[1]{\paragraph{#1}\mbox{}\\}

\title{\textsc{Design description of the model}}
\date{\today}

\begin{document}

\begin{titlepage}
	\centering
	{\scshape\LARGE Design description of the model \par}
	\vspace{1cm}
	{\scshape\Large Rev 0.1.0 \par}
	\vspace{2cm}
	\vfill

% Bottom of the page
	{\large \today\par}
\end{titlepage}

\thispagestyle{empty}
\newpage

\setcounter{tocdepth}{3}
\tableofcontents
\setcounter{secnumdepth}{3}
\newpage

\section{Vessel model}

\begin{equation}
\label{eq:vessel_model}
\begin{aligned}
	\boldsymbol{\dot{\eta}} &= \boldsymbol{R} (\psi) \boldsymbol{\nu} \\
	(\boldsymbol{M}_{RB} + \boldsymbol{M}_A) \boldsymbol{\dot{\nu}} + \boldsymbol{D} \boldsymbol{\nu} |\boldsymbol{\nu}| &= \boldsymbol{\tau}_{thr} + \boldsymbol{\tau}_{wind} + \boldsymbol{\tau}_{current} + \boldsymbol{\tau}_{ext}
\end{aligned}
\end{equation}
\\
\\
The matrix $\boldsymbol{M}_{RB}$ can be defined as
\begin{equation}
\label{eq:MRB_matrix}
\begin{aligned}
	\boldsymbol{M}_{RB} &=
		\left[ \begin{array}{ccc}
			m &  0 & 0 \\
			0 &  m & 0 \\
			0 &  0 &  I_z 
		\end{array} \right],
\end{aligned}
\end{equation}
%
where $m$ is the displacement, i.e. the mass of the displaced fluid, or the mass of the vessel, and $I_z$ is the moment of inertia about the $z_b$-axis. The elements that are not on
the diagonal of the matrix are ignored.
\\
\\
The added-mass matrix, $\boldsymbol{M}_A$, is calculated in the origin of the coordinate system, CO. This matrix can be written as

\begin{equation}
\label{eq:MA_matrix}
\begin{aligned}
	\boldsymbol{M}_A &=
		\left[ \begin{array}{ccc}
			-X_{\dot{u}} &  0 & 0 \\
			0 &  -Y_{\dot{v}} & 0 \\
			0 & 0 &  -N_{\dot{r}} 
		\end{array} \right],
\end{aligned}
\end{equation}
%
in SNAME notation. $X_{\dot{u}}$ is added mass in surge, $Y_{\dot{v}}$ is added mass in sway og $N_{\dot{r}}$ is added mass ini yaw. The elements that are not on
the diagonal of the matrix are ignored.
\\
\\
The matrix $\boldsymbol{D}$ can, in SNAME notation, be written as

\begin{equation}
\label{eq:D_matrise}
\begin{aligned}
	\boldsymbol{D} &=
		\left[ \begin{array}{ccc}
			-X_u &  0 & 0 \\
			0 &  -Y_v & 0 \\
			0 & 0 &  -N_r 
		\end{array} \right],
\end{aligned}
\end{equation}
%
where $X_u$ is drag in surge, $Y_v$ is drag in sway and $N_r$ is drag in yaw. The elements that are not on the diagonal of the matrix are ignored.
\\
\\
$\boldsymbol{\tau}_{thr} = [\tau_{thr,X}, \tau_{thr,Y}, \tau_{thr,N}]^{\top}$ are forces from thrusters in surge, sway and yaw. 
$\boldsymbol{\tau}_{wind} = [\tau_{wind,X}, \tau_{wind,Y}, \tau_{wind,N}]^{\top}$ and $\boldsymbol{\tau}_{ext} =
 [\tau_{ext,X}, \tau_{ext,Y}, \tau_{ext,N}]^{\top}$ are wind forces and external forces (pipe, winch, etc.) that affects the vessel.
\\
\\
Rotation from vessel coordinates (BODY) to Earth coordinates (NED) can be done with a rotation matrix, $\boldsymbol{R}(\psi)$. For three degrees of freedom, this can be written as

\begin{equation}
\label{eq:rotation_matrix}
\begin{aligned}
	\boldsymbol{R}(\psi) &=
		\left[ \begin{array}{ccc}
			\cos(\psi) &  -\sin(\psi) & 0 \\
			\sin(\psi) &  \cos(\psi) & 0 \\
			0 & 0 &  1 
		\end{array} \right],
\end{aligned}
\end{equation}
%
where $\psi$ is the heading of the vessel. $\boldsymbol{\eta} = [N, E, \psi]^{\top}$ is the position in North, East and heading. $\boldsymbol{\nu} = [u, v, r]^{\top}$ 
is velocity in surge, sway and yaw.


\section{Thruster model}

The force that a single thruster can use can be written as

\begin{equation}
	T = K_T \rho D^4 n^2,
\end{equation}
%
where $T$ is the thruster force, $\rho$ is the density of water, $D$ is the diameter of the propeller and $n$ is the rpm of the propeller. $K_T$ is an empirical value that is
dependent on water speed into the propeller and the pitch angle of the propeller. A simplified version can be written as

\begin{equation}
	K_T = K \cdot \theta^\alpha,
\end{equation}
%
where $K$ is a constant, $\theta$ is pitch angle normalized to 0 $\rightarrow$ 1 (0\% $\rightarrow$ 100\%) and $\alpha$ is a constant. Because $K$, $\rho$ og $D$ are constants
they can be merged together into one constant. $n$ can also be normalized such that it's between $0$ and $1$: $n_n = K_n \cdot n$. The final expression for $T$ is then

\begin{equation}
\begin{aligned}
	T &= K_T \rho D^4 n^2 \\
	   &= K \theta^\alpha D^4 (K_n n_n)^2 \\
	   &= K \theta^\alpha D^4 K_n^2 n_n^2 \\
	   &= T_{K} \cdot \theta^\alpha n_n^2,
\end{aligned}
\end{equation}
%
where $T_{K} = K D^4 K_n^2$.
%
This model is suitable for bollard pull condition, i.e. zero water speed through the propeller other than the self-induced speed.

\section{Wind model}

$\boldsymbol{\tau}_{wind}$  are wind forces that influence the vessel. The forces can be written as:

\begin{equation}
\label{eq:vindkrefter}
\begin{aligned}
	\boldsymbol{\tau}_{wind} &=
		\left[ \begin{array}{ccc}
			q \cdot C_X \cdot  A_f \\
			q  \cdot C_Y \cdot A_l \\
			q  \cdot C_N \cdot A_f \cdot Loa
		\end{array} \right],
\end{aligned}
\end{equation}
%
where $C_X$, $C_Y$ og $C_N$ are wind coefficients (drag coefficients) in surge, sway og yaw, $A_f$ is projected frontal area and $A_l$ is projected lateral area.
$q = \frac{1}{2} \cdot \rho_{air} \cdot V_{w,r}^2$, where $\rho_{air}$ is the density of air and $V_r$ is relative wind velocity.
$C_X$, $C_Y$ og $C_N$ are typically found with emprical methods such as Blendermann. These are functions of relative wind velocity.

\section{Current model}

\end{document}