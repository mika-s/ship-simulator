\documentclass[a4paper]{article}  

\usepackage{amsmath}
\usepackage{amssymb}
\usepackage{mathrsfs}
\usepackage[toc,page]{appendix}
\usepackage[pdftex]{graphicx}
\usepackage{geometry}
\usepackage[utf8]{inputenc}
\usepackage{tabularx}
\usepackage{color}
\usepackage{natbib}
\usepackage{epstopdf}
\usepackage{caption}
\usepackage{subcaption}
\usepackage{cancel}
\usepackage{pdfpages}

\bibliographystyle{plainnat}
\newcommand{\HRule}{\rule{\linewidth}{1mm}}
\newcommand{\myparagraph}[1]{\paragraph{#1}\mbox{}\\}

\title{\textsc{Design description of the controllers}}
\date{\today}

\begin{document}

\begin{titlepage}
	\centering
	{\scshape\LARGE Design description of the controllers\par}
	\vspace{1cm}
	{\scshape\Large Rev 0.1.0 \par}
	\vspace{2cm}
	\vfill

% Bottom of the page
	{\large \today\par}
\end{titlepage}

\thispagestyle{empty}
\newpage

\setcounter{tocdepth}{3}
\tableofcontents
\setcounter{secnumdepth}{3}
\newpage

\section{Autopilot}

There are two parameters to control during autopilot: surge speed and heading.

\subsection{Heading}

The heading of the vessel in autopilot is currently controlled by a PID controller.

\subsubsection{PID}

The basic equation for the PID controller can be written as

\begin{equation}
\label{eq:heading_pid}
\begin{aligned}
	\tau_{\psi} = K_p \cdot e + K_i \cdot \int e + K_d \cdot \dot{e}
\end{aligned}
\end{equation}
%
where $\tau_{\psi}$ is wanted heading force, $K_p$ is the proportional gain, $K_i$ is the integral gain and $K_d$ is the derivative gain. $e$ is the error and is defined as
\begin{equation}
\label{eq:heading_pid_error}
\begin{aligned}
	e = \psi_d - \psi
\end{aligned}
\end{equation}
%
where $\psi_d$ is the desired heading.

\paragraph{Anti-windup}

The I-term can increase to very large or decrease to very small numbers. This will cause significant overshoot when reaching the setpoint during a heading change, for example. 
A couple of mechanisms are implemented to prevent the I-term from winding up.

\begin{itemize}  
\item The I-term doesn't start building up unless the vessel's heading is within a sector of the desired heading. This sector is set in Controller.json and goes to both positive and negative
direction around the desired heading.

\item The I-term can never go above a value given in Controller.json.

\item The I-term starts building down when within iDieSector (close to the desired heading). It goes to zero with a constant of iDieConstant per second.
\end{itemize}

\subsection{Speed}

\subsubsection{PID}


\end{document}